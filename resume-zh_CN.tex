% !TEX TS-program = xelatex
% !TEX encoding = UTF-8 Unicode
% !Mode:: "TeX:UTF-8"

\documentclass[10pt]{resume}
\usepackage{zh_CN-Adobefonts_external} % Simplified Chinese Support using external fonts (./fonts/zh_CN-Adobe/)
% \usepackage{NotoSansSC_external}
% \usepackage{NotoSerifCJKsc_external}
% \usepackage{zh_CN-Adobefonts_internal} % Simplified Chinese Support using system fonts
\usepackage{linespacing_fix} % disable extra space before next section
\usepackage{cite}

\begin{document}
\pagenumbering{gobble} % suppress displaying page number
\footnotesize

\name{张淇}

\basicInfo{
  \email{copyrightpoi@icloud.com} \textperiodcentered\ 
  \phone{(+86) 173-6403-9183} \textperiodcentered\ 
 \faGithub ~ github.com/copyrightpoiiiii}
 
 
\section{\faGraduationCap\  教育背景}

\datedsubsection{\textbf{华中科技大学}, 武汉}{2021 -- 至今}
\textit{在读研究生}\ 计算机科学与技术学院

\datedsubsection{\textbf{华中科技大学}, 武汉}{2017 -- 2021}
\textit{学士}\ 计算机科学与技术学院(卓越工程师班)

\textit{在校成绩}\  前25\% , \ CET-6,\ CSP成绩: 330

\section{\faUsers\ 项目经历}
\datedsubsection{\textbf{分组染色问题} }{2018年6月 -- 2018年11月}
\role{C++, 图论, 组合优化}{个人项目}
\ \ \ \ 设计了一种新的求解组合染色问题的启发式算法

\ \ \ \ 结合了遗传算法,邻域搜索和禁忌搜索进行求解,并且将分组染色问题的求解和传统的图染色算法求解连接了起来,同时打破了部分算例的运行时间纪录


\datedsubsection{\textbf{MiniSat}}{2018年11月 -- 2019年3月}
\role{C++, 组合优化}{个人项目}
\begin{onehalfspacing}
\ \ \ \ 使用C++实现了MiniSat求解器,可以求解命题可满足性问题

\ \ \ \ 根据MiniSat的相关论文实现了MiniSat算法,在实现过程中还实现了多种数据结构模板:Vector, Priority\_Queue, Set
\end{onehalfspacing}

\datedsubsection{\textbf{网站服务器}}{2019 年11月 -- 2019 年11月}
\role{C++,Qt,MacOS}{个人项目}
\ \ \ \ 一个可以运行在Mac系统的具有图形化界面的网站服务器

\ \ \ \ 该服务器使用Qt实现图形化界面, 使用kqueue实现多线程调度,并且可以配置网站地址,端口号和工作目录

\datedsubsection{\textbf{数据库在CI环境的容器化}}{2020 年10月 -- 2021年1月}
\role{微软中国(北京)}{C\#,Docker}
\begin{onehalfspacing}
\ \ \ \ 构建Pipeline自动将某业务后台数据库构建为Docker镜像

\ \ \ \ 在CI环境使用构建好的Docker镜像进行测试

\ \ \ \ 采用此方式后,CI环境初始化阶段的数据传输量下降45\%,CI Pipeline运行速度提升10\%


\end{onehalfspacing}

% Reference Test
%\datedsubsection{\textbf{Paper Title\cite{zaharia2012resilient}}}{May. 2015}
%An xxx optimized for xxx\cite{verma2015large}
%\begin{itemize}
%  \item main contribution
%\end{itemize}

\section{\faCogs\ IT 技能}
% increase linespacing [parsep=0.5ex]
\begin{itemize}[parsep=0.5ex]
  \item 编程语言: C > C++ > C\# > Java
  \item 平台: Linux
  \item 开发: Web, Sql , Docker
\end{itemize}

\section{\faHeartO\ 获奖情况}
\datedline{\textit{银牌}, 2018年湖北省大学生程序设计竞赛}{2018 年4 月}
\datedline{\textit{银牌}, CCF大学生计算机系统与程序设计大赛(CCSP) }{2018年10 月}
\datedline{\textit{一等奖}, 蓝桥杯 }{2019月}

%\section{\faInfo\ 其他}
% increase linespacing [parsep=0.5ex]
%\begin{itemize}[parsep=0.5ex]
%  \item CSP成绩: 330
%  \item 语言: 英语 - 六级
%\end{itemize}

%% Reference
%\newpage
%\bibliographystyle{IEEETran}
%\bibliography{mycite}
\end{document}